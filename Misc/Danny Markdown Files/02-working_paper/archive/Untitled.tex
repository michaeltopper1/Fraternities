% Options for packages loaded elsewhere
\PassOptionsToPackage{unicode}{hyperref}
\PassOptionsToPackage{hyphens}{url}
%
\documentclass[
]{article}
\usepackage{lmodern}
\usepackage{amssymb,amsmath}
\usepackage{ifxetex,ifluatex}
\ifnum 0\ifxetex 1\fi\ifluatex 1\fi=0 % if pdftex
  \usepackage[T1]{fontenc}
  \usepackage[utf8]{inputenc}
  \usepackage{textcomp} % provide euro and other symbols
\else % if luatex or xetex
  \usepackage{unicode-math}
  \defaultfontfeatures{Scale=MatchLowercase}
  \defaultfontfeatures[\rmfamily]{Ligatures=TeX,Scale=1}
\fi
% Use upquote if available, for straight quotes in verbatim environments
\IfFileExists{upquote.sty}{\usepackage{upquote}}{}
\IfFileExists{microtype.sty}{% use microtype if available
  \usepackage[]{microtype}
  \UseMicrotypeSet[protrusion]{basicmath} % disable protrusion for tt fonts
}{}
\makeatletter
\@ifundefined{KOMAClassName}{% if non-KOMA class
  \IfFileExists{parskip.sty}{%
    \usepackage{parskip}
  }{% else
    \setlength{\parindent}{0pt}
    \setlength{\parskip}{6pt plus 2pt minus 1pt}}
}{% if KOMA class
  \KOMAoptions{parskip=half}}
\makeatother
\usepackage{xcolor}
\IfFileExists{xurl.sty}{\usepackage{xurl}}{} % add URL line breaks if available
\IfFileExists{bookmark.sty}{\usepackage{bookmark}}{\usepackage{hyperref}}
\hypersetup{
  pdftitle={Shrinkage Among Time Varying Weights in Counterfactual Analysis},
  pdfauthor={Danny Klinenberg},
  hidelinks,
  pdfcreator={LaTeX via pandoc}}
\urlstyle{same} % disable monospaced font for URLs
\usepackage[margin=1in]{geometry}
\usepackage{graphicx,grffile}
\makeatletter
\def\maxwidth{\ifdim\Gin@nat@width>\linewidth\linewidth\else\Gin@nat@width\fi}
\def\maxheight{\ifdim\Gin@nat@height>\textheight\textheight\else\Gin@nat@height\fi}
\makeatother
% Scale images if necessary, so that they will not overflow the page
% margins by default, and it is still possible to overwrite the defaults
% using explicit options in \includegraphics[width, height, ...]{}
\setkeys{Gin}{width=\maxwidth,height=\maxheight,keepaspectratio}
% Set default figure placement to htbp
\makeatletter
\def\fps@figure{htbp}
\makeatother
\setlength{\emergencystretch}{3em} % prevent overfull lines
\providecommand{\tightlist}{%
  \setlength{\itemsep}{0pt}\setlength{\parskip}{0pt}}
\setcounter{secnumdepth}{5}

\title{Shrinkage Among Time Varying Weights in Counterfactual Analysis}
\author{Danny Klinenberg}
\date{Last Updated: 2020-06-22}

\begin{document}
\maketitle
\begin{abstract}
I plan to introduce shrinkage to time varying parameters in a synthetic
control framework using state space models. I plan to do this using the
Bayesian Lasso. Past state space models focus on shrinkage of the point
estimate. This means shrinkage on the constant part of the coefficients.
I introduce models that include shrinkage on the time-varying part of
the coefficeints to the synthetic control framework. I referred to these
models as Time Varying Parameter Shrinkage (TVPS) state space models. I
plan to conduct a simulation study comparing TVPS state space models to
traditioinal synthetic control. My objective is to demonstrate this
class of shrinkage estimators perform just as well as synthetic control
in a traditional synthetic control scenario as well as in situations
where synthetic controls assumptions are not met. My contribution is the
introduction of TVPS state space models to the synthetic control
framework.
\end{abstract}

\hypertarget{introduction}{%
\section{Introduction}\label{introduction}}

here I am \texttt{3} in a line Synthetic control is a panel data
approach to analyze the effect of a treatment on very few treated units.
In many cases, there is only one treated unit. Uses of synthetic control
have included the effect of joining the European Union on economic
growth (Campos, Coricelli, and Moretti 2019a) and effect of terrorism on
GDP (Abadie and Gardeazabal 2003). The intuitive idea behind synthetic
control methods is to construct a counterfactual for the treated
observations using a weighted average of the untreated. If the
constructed, or synthetic, observation fits the treated observation
``well before'' the intervention, then the synthetic observation is said
to have worked and qualifies as a counterfactual. ``Fits the treated
observation'' means the synthetic control is almost perfectly matching
in the pretreatment. ``Well before'' has been defined differently by
different authors, but a general consensus is at least ten periods. A
causal interpretation can then be made on the effect of the treatment in
following periods by comparing the synthetic observation to the actual.

Causal inference in a synthetic control framework is jeopardized when
the weights used for the counterfactual are non-constant. This means
that the synthetic control does not ``fit the treated observation''.
Common diagnostic checks on synthetic control output should alert the
researcher to violations. The problem then becomes many answers to
important questions are unattainable. One solution is to explicitely
account for the changing relationship using dynamic coefficients in
state space models. The addition of changing weights can lead to the
synthetic control ``fitting well'' while the constant weights
assumptions is violated. However, using all dynamic coefficients will
lead to overfitting and implausibly large probability intervals
(Brodersen et al. 2015a). Recent developments in time varying parameter
state space models have introduced shrinkage estimators to reduce
dynamic coefficients to static ones in the event of overfitting. Time
varying parameter shrinkage (TVPS) state space models have been utilized
in macroeconometrics for inflation predictions and stock returns, but
not yet synthetic control ((Belmonte, Koop, and Korobilis 2014) and
(Frühwirth-Schnatter and Wagner 2010)). My contribution is introducing
this class of models to the synthetic control framework. The
introduction will be done through a simulation study.

\hypertarget{working-literature-review}{%
\section{Working Literature Review}\label{working-literature-review}}

\hypertarget{current-state-of-counterfactual-analysis-adh-synthetic-control}{%
\subsection{Current State of Counterfactual Analysis: ADH Synthetic
Control}\label{current-state-of-counterfactual-analysis-adh-synthetic-control}}

In many research endeavors, a true counterfactual does not exist. In
addition, there are relatively few treated observation. This means a
researcher cannot use traditional tools such as difference in
differences, regression discontinuity, or simple randomization.
Synthetic Control was proposed as an alternative in which observations
unaffected by a policy are pooled together to create a counterfactual
for the treated observations. This was first popularly employed in
(Abadie and Gardeazabal 2003) and formalized in a followup paper
(Abadie, Diamond, and Hainmueller 2010) (referred to as \emph{ADH
Synthetic Control}). If there exists constant weights bounded between 0
and 1 that sum to unity such that a weighted average of control
observations match the treated observation's covariates well in
pre-treatment periods, then that average is used as a counterfactual in
the post treatmen periods. ADH is a powerful too that has spurred a
whole line of literature.

Recent developments in synthetic control have focused on three goals:
the bounded assumption, adding more treated observations, and matching
on covariates. (Doudchenko and Imbens 2016) first noted that ADH
synthetic control mirror machine learning processes. They then suggested
using an alternative machine learning process, elastic net, to produce
counterfactuals. Similarly, (Athey et al. 2018) approached the synthetic
control problem from a machine learning perspective using matrix
completion methods. The idea viewed developing a counterfactual as a
missing data problem in a matrix. Both method do not require the weights
sum to unity and be bounded. (Xu 2017) and ({\textbf{???}}) generalized
ADH synthetic controls to include many treated observations with
treatment heterogeneity and start dates. Finally, ({\textbf{???}})
compared creating weights through matching on covariates (ADH Synthetic
Control) to only matching on the outcome. This paper showed that
matching on covariates provided little benefit to estimation.

Suppose there is an outcome of interest, \(Y_{it}\). Suppose there are
N+1 observations observed over T years with one observation being
exposed to treatment. Without loss of generality, assume observation
\(i=1\) is treated starting at time \(T_0\). Furthermore, define the
treatment \emph{D} such that \(D_{1t}=1*I(t \ge T_0)\) with treatment
effect \(treat_{1,t}\). Using the potential outcomes framework, assume
there are two states of the world: untreated \(Y_{it}(0)\) and treated
\(Y_{it}(1)=treat_{it} D_{it}+Y_{it}(0)\). The objective is to measure:

\[treat_{1t}=Y_{1t}(1)-Y_{1t}(0)\]

Furthermore, assume that \(Y_{it}(0)\) comes from the following data
generating process:

\[Y_{it}(0)=\delta_t+\theta_t Z_i +\lambda_t \mu_i+\epsilon_{it}\] where
\(\delta_t\) is an unknown common time factor constant across units,
\(\theta_t\) is a (1 x r) of unknown parameters, \(Z_t\) is a (r x 1)
vector of observed covariates that are not affected by treatment,
\(\lambda_t\) is a (1 x F) vector of unobserved common factors,
\(\mu_i\) is a (F x 1) vector of unobserved factor loadings, and
\(\epsilon_{it}\) is the error term.

Unfortunately, \(Y_{1t}(1)\) and \(Y_{1t}(0)\) are never observed at the
same time. Therefore, a counterfactual must be created. This is done
through a convex set of the control observations over the pre-treatment
period. The control observations are referred to as the \emph{donor
pool}. Formally, the donor pool is defined as
\(Y_{jt}(0) \forall j \ne 1\). If a study involved the policy
implications on a country's GDP, the control observations would be other
countries' GDP.

Suppose there exists a weighting matrix \(W=(w_2,...,w_{n+t})'\) such
that \(w_i \in [0,1]\) and \(\sum_{i=2}^{n+1} w_i=1\). ADH bounds the
weights between 0 and 1 and sums them to unity to avoid extrapolation
bias\footnote{This assumption has been thoroughly investigated. There
  are countless papers questioning this assumption. A good starting
  point is (Doudchenko and Imbens 2016).}. Choosing an optimal weighting
matrix, \(w^*\), would create an unbiased estimator. Namely:

\[\boxed{
\begin{aligned}
\text{ADH Assumption: The }& \text{estimator of $Y_{i1}(0)$ is unbiased if}:\\
&\sum_{i=2}^{n+1}w_i^*Z_i=Z_1\\
&\sum_{i=2}^{n+1}w_i^* \mu_i =\mu_1
\end{aligned}
}\]

Using \(w^*\) would yield:

\[
\begin{aligned}
Y_{1,1}(0)&=\sum_{i=2}^{n+1} w^*_i Y_{i,1}(0)\\
Y_{1,2}(0)&=\sum_{i=2}^{n+1} w^*_i Y_{i,2}(0)\\
&.\\
&.\\
&.\\
Y_{1,T_0}(0)&=\sum_{i=2}^{n+1} w^*_i Y_{i,T_0}(0)
\end{aligned}
\]

The equality above implies that the weighted average of the other
observations creates a perfect counterfactual for the treated unit. On
an intuitive level, the notion of constant weights means the
relationship between the treated unit and controls is constant
throughout the analysis period. If a synthetic control analysis is being
employed to study country GDP (as is quite common), each country cannot
have a significant change throughout the analysis. This means no major
governmental shifts, policy reforms, or economic booms/busts. Given that
the synthetic control literature recommends a long pre-treatment window
of ten periods and post of five, countries cannot undergo major changes
in 15 periods. Since most country level data is reported at the yearly
level, this translates to a 15 year constant relationship between the
treatment and control.

Mathematically, the intuition is describing a situation where \(w^*\)
holds with noise. This means
\(Y_{1,t} \ne \sum_{i=2}^{n+1} w^*_i Y_{i,t}(0)\) for some t. This would
be a violation of the ADH assumptions. One way to account for this error
is to allow time specific weights:
\(Y_{1,t} = \sum_{i=2}^{n+1} w^*_{i,t} Y_{i,t}(0)\). An immediate
concern with this strategy is identification. Instead of having \(T_0\)
equations and n coefficients (e.g.~\([w^*_2,...w^*_{n+1}]\)), now there
is 1 equation with n coefficients
(e.g.~\([w^*_{2,t},...,w^*_{n+1,t}]\)). Another immediate concern of
using all time varying coefficients is overfitting ((Bitto and
Frühwirth-Schnatter 2019), (Belmonte, Koop, and Korobilis 2014)).
Overfitting is when the model too closely fits the data rather than
uncovering the true data generating process. This leads to erroneous out
of sample predictions. In a Bayesian framework, it also leads to
impractically large prediction intervals (Scott and Varian, n.d.). To
account for the danger of overfitting, shrinkage can be applied.
Shrinkage in a regression is when coefficients are biased towards zero
to decrease mean squared error at the expense of increased bias.
Shrinkage estimators have gained extreme popularity in statistic and
machine learning. I refer to models that apply shrinkage to time varying
coefficients as Time Varying Parameter Shrinkage (TVPS) models. I plan
to apply TVPS models to a synthetic control framework.

\hypertarget{linear-gaussian-state-space-modeling}{%
\subsection{Linear Gaussian State Space
Modeling}\label{linear-gaussian-state-space-modeling}}

The problem described above is a latent variable estimation problem.
State space modeling is a time series concept that allows for modeling
latent variables. This means modeling unobserved components like time
trends, seasonality, and time varying coefficients. Thinking back to
synthetic control, this would be the elements of \(\mu_i\). A state
space model is composed of an observation equation and transition
equation. A general form of these equations are:

\[
\begin{aligned}
y_t&=Z_t\alpha_t+\epsilon_t & \text{observation equation}\\
\alpha_{t+1}&=T_t \alpha_t +R_t \eta_t & \text{transition equation}\\
\alpha_0 &\sim N(a_0, P_0)
\end{aligned}
\] where \(\epsilon_t \sim N(0,\sigma_t^2)\) and
\(\eta_t \sim N(0,Q_t)\) are independent of all unknown factors. \(y_t\)
is the observed data and \(\alpha_t\) is a combination of observed data
(e.g.~control variables) and unobserved components (e.g.~trend and
cycle). In the case of a scalar output, \(y_t\), with \(m\) variables
and \(r\) time varying components, \(Z_t\) would be a 1 x m dimensional
matrix, \(\alpha_t\) a m x 1 matrix, and \(\epsilon_t\) a scalar.
\(\alpha_{t+1}\) would also be a m x1 matrix, \(T_t\) an m x m matrix,
\(R_t\) a m x r matrix and \(Q_t\) and r x r matrix. Finally, \(a_0\) is
m x 1 and \(P_0\) is m x m. Linear Gaussian state space models are
structural models. The assumptions necessary for linear Gaussian state
space models are:

\begin{enumerate}
\def\labelenumi{\arabic{enumi})}
\item
  \(\epsilon_t \sim N(0,\sigma^2_t)\) and \(\eta_{t} \sim N(0, Q_t)\).
  The errors are also assumed to be mutually and serially uncorrelated.
  This is because they are meant to be random disturbances within the
  model.
\item
  The errors must be normal.
\item
  the transition equations can be of lag order 1. Any additional lag
  orders can be rewritten as order 1 using the state space framework.
\end{enumerate}

(Scott and Varian, n.d.) proposed using state space models in
combination with Spike and Slab variable selection for high dimensional
time series analysis. This was then extended by (Brodersen et al. 2015a)
as an alternative for synthetic control. Both (Scott and Varian, n.d.)
and (Brodersen et al. 2015a) warn about the use of time varying
coefficients in their methods. Their method focuses on shrinkage among
the coefficients only. This means the shrinkage determines if a variable
should be included or not. This method does not differentiate between
static and dynamic inclusion. This is set exogenously by the researcher.
Assuming all relevant variables are dynamic leads to both overfitting
and implausibly large probability intervals.

Recent developments in time varying parametric estimation has extended
shrinkage to both the static and dynamic portion of coefficients. This
means coefficients are biased towards being static or irrelevant,
similarly to frequentest shrinkage. I apply (Belmonte, Koop, and
Korobilis 2014) method to synthetic control. The method focused on using
the Bayesian Lasso first proposed by (Park and Casella 2008) to shrink
the static and dynamic portion of coefficients. This is a direct
extension of ADH synthetic control: (Kinn 2018) argued synthetic control
is a restricted version of Lasso similarly to how (Doudchenko and Imbens
2016) compared elastic net to synthetic control. Bayesian Lasso is
identical to Lasso under independent Laplace priors (Park and Casella
2008). This methodology applies Bayesian Lasso to time varying
parameters.

\hypertarget{the-model}{%
\section{The Model}\label{the-model}}

\hypertarget{initial-setup}{%
\subsection{Initial Setup}\label{initial-setup}}

I plan to use the model proposed by (Belmonte, Koop, and Korobilis
2014). Suppose the state space model is defined as:

\begin{align}
y_{1,t}&= \sum_{j=2}^{J+1} \beta_{j,t}y_{j,t}+\epsilon_t & \epsilon_t|\sigma^2 \sim N(0, \sigma^2) &\\
\beta_{j,t}&=\beta_{j,t-1}+\eta_{j,t} & \eta_{j,t} \sim N(0,\theta_j) &\ \ \ \  \forall j\\
\beta_{j,0}&\sim N(\beta_j,\theta_j P_{jj}) & &\ \ \ \ \forall j
\end{align}

where \(P_{jj}\) is a hyperparameter. This specification of \(\theta_j\)
lends itself to a useful interpertation: \(\theta_j\) govern the
dynamics of \(\beta_{j,t}\).

The errors are assumed to be independent of one another and independent
of all leads and lags. Notice, I am also assuming the errors between
coefficients are independent (e.g.~\(cov(\eta_{j,t},\eta_{i,t})=0\) for
\(i\ne j\)). This assumption is to keep the model relatively
parsimonious ((Belmonte, Koop, and Korobilis 2014), (Bitto and
Frühwirth-Schnatter 2019)). If \(\theta_j=0\), then \(\beta_j\) is a
static coefficient. Past counterfactual analysis papers have introduced
shrinkage to \(\beta_j\) in a synthetic control framework
(e.g.~(Brodersen et al. 2015a)). This paper will add shrinkage to
\(\theta_j\) in that framework.

The transition equation can be rewritten to decompose \(\beta_j\) into a
time varying and constant components ((Frühwirth-Schnatter and Wagner
2010)).

\begin{align}
\beta_{j,t}&=\beta_{j}+\tilde{\beta}_{j,t}\sqrt{\theta_j}\\
\tilde{\beta}_{j,t}&= \tilde{\beta}_{j,t-1}+\tilde{\eta}_{j,t} & \tilde{\eta}_{j,t} \sim N(0,1)\\
\tilde{\beta}_{j,0}& \sim N(0,P_{jj})
\end{align}

To verify these are equal, note:

\[
\begin{aligned}
\beta_{j,t}-\beta_{j,t-1}&=(\beta_j + \sqrt\theta_j \tilde{\beta}_{j,t})-(\beta_j + \sqrt{\theta_j}\tilde{\beta}_{j,{t-1}}) & \text{Plugging in (4)}\\
& =\sqrt{\theta_j}(\tilde{\beta}_{j,t}-\tilde{\beta}_{j,t-1}) & \text{Regroup}\\
&= \sqrt{\theta_j}(\tilde{\beta}_{j,t-1}+\tilde{\eta}_{j,t}-\tilde{\beta}_{j,t-1}) & \text{Plug in (5)}\\
&= \sqrt{\theta_j}\tilde{\eta}_{j,t} & \text{Simplify}\\
\end{aligned}
\] Notice that \(\tilde{\eta}_{j,t} \sim N(0,1)\). Therefore
\(\sqrt{\theta_j}\tilde{\eta}_{j,t} \sim N(0,\theta_j)\) which is
\(\eta_{j,t}\). This shows that equation (2) and (5) are equivilant.

\(\beta_j\) can now be interpreted as the point estimate of
\(\beta_{j,t}\) and \(\sqrt{\theta_j}\) the time varying portion. The
advantage of this formulation is that shrinkage estimation can be
performed on both aspects of the coefficient. Plugging the reformulation
back into the original equation yields the state space model:

\[
\begin{aligned}
y_{1,t}&= \sum_{j=2}^{J+1} \left(\beta_{j}+\tilde{\beta}_{j,t}\sqrt{\theta_j}\right)y_{j,t}+\epsilon_t & \epsilon_t|\sigma^2 \sim N(0, \sigma^2)\\
\tilde{\beta}_{j,t}&= \tilde{\beta}_{j,t-1}+\tilde{\eta}_{j,t} & \tilde{\eta}_{j,t} \sim N(0,1)\\
\tilde{\beta}_{j,0}& \sim N(0,P_{jj})
\end{aligned}
\]

(Frühwirth-Schnatter and Wagner 2010) refer to this setup as the
\emph{non-centered parameterization of state space models}. This is
extremely useful because the problem of \emph{variance selection} has
now been recast as one of \emph{variable selection}. Variable selection
problems are far better understood and applied. However, there are some
additional precautions that must be made when working with non-centered
parameterization of state space models. One such issue is an
identification problem arises in that
\(\sqrt{\theta_j}\tilde{\beta_{j,t}}\) can be replaced by
\((-\sqrt{\theta_j})(-\tilde{\beta_{j,t}})\) without affecting the
likelihood function. The appropirate solution to this issue is discussed
in (Frühwirth-Schnatter and Wagner 2010) and implemented in the Gibbs
sampler.

The setup allows for four possibilities:

\begin{enumerate}
\def\labelenumi{\roman{enumi})}
\item
  constant coefficient (\(\beta_j\) not shrunk to 0 but
  \(\sqrt{\theta_j}\) shrunk to 0)
\item
  irrelevant coefficient (\(\beta_j\) shrunk to 0 and
  \(\sqrt{\theta_j}\) shrunk to 0)
\item
  small time-varying coefficient (\(\beta_j\) shrunk to 0 but
  \(\sqrt{\theta_j}\) not shrunk to 0)
\item
  time-varying coefficient (\(\beta_j\) not shrunk to 0 and
  \(\sqrt{\theta_j}\) not shrunk to 0)
\end{enumerate}

\hypertarget{the-priors}{%
\subsection{The Priors}\label{the-priors}}

The goal of this paper is to utilize Bayesian Lasso because of it's
direct relationship to ADH synthetic control. Recall that (Park and
Casella 2008) showed the Bayesian Lasso is identical to Lasso under
independent LaPlace priors which is identical to a mean 0 normal
distribution with variance defined as exponential. This direct
relationship to ADH synthetic control provides a nice segway into the
TVPS state space model literature. Following in suit, I choose:

\[
\begin{aligned}
\bf{\beta} | \tau^2, \sigma^2 &\sim N(0,\sigma^2 diag[\tau_2^2,...\tau_J^2])\\
\tau_j^2 | \lambda^2 &\sim exp\left(\frac{\lambda^2}{2}\right)
\end{aligned}
\] \(\lambda^2\) can then be calculated via MLE or with it's own prior.
Staying true to (Park and Casella 2008), I choose
\(\lambda^2 \sim Gamma(b_1,b_2)\), with \(b_1\) and \(b_2\)
hyperparameters to be set.

Traditionally, variances have been defined by the inverse gamma
distribution. However, the inverse gamma does not allow for effective
shrinkage given it's support. (Frühwirth-Schnatter and Wagner 2010)
provide an in depth argument for the use of the normal distribution as
an alternative. Briefly, inverse gammas perform poorly in terms of
shrinkage and the normal distribution does not. Similarly to
\(\beta_j\), I define \(\sqrt{\theta_j}\) as:

\[
\begin{aligned}
\sqrt{\theta} | \xi^2 &\sim N(0, \sigma^2 diag[\xi_2^2,...\xi_J^2])\\
\xi_j^2 | \kappa^2 &\sim exp\left(\frac{\kappa^2}{2}\right)
\end{aligned}
\] where \(\kappa^2 \sim Gamma(c_1,c_2)\) with \(c_1\) and \(c_2\) as
hyperparameters.

All that's left is to define \(\sigma^2\). I set
\(\frac{1}{\sigma^2} \sim Gamma(a_1,a_2)\) with \(a_1\) and \(a_2\) as
hyperparameters. Again, the purpose for defining the priors as such is
to recreate the Bayesian Lasso. The Bayesian Lasso is identical to the
frequentist Lasso with the priors described above and ADH synthetic
control is a restricted version of Lasso (Kinn 2018).

\hypertarget{identifying-assumptions}{%
\section{Identifying Assumptions}\label{identifying-assumptions}}

Up until this point, this paper has focused on the estimation technique.
In order to gain causal inference, two assumptions must be implemented.
These are not the only assumptions being made within the model. Every
prior choice, hyperparameter, and state space formulation are also
assumptions being made. However, the following two assumptions are
necessary for the results from TVPS state space model to be causal:

\begin{enumerate}
\def\labelenumi{\roman{enumi})}
\item
  The control time series are unaffected by the treatment. If his were
  to be violated, the causal estimates would be biased.
\item
  The dynamic relationship between the treated variable and the controls
  established in the pretreatment periods does not change.
\end{enumerate}

\hypertarget{monte-carlo-simulation-data}{%
\section{Monte Carlo Simulation
Data}\label{monte-carlo-simulation-data}}

The simulation is restricted to the outcomes of the observed units,
without considering underlying covariates. A growing body of literature
has supported synthetic control analysis without covariates. (Athey and
Imbens 2017) and (Doudchenko and Imbens 2016) argue the outcomes tend to
be far more important than covariates in terms of predictive power. They
further argue that minimizing the difference between treated outcomes
and control outcomes prior to treatment tend to be sufficient to
construct a synthetic control. ({\textbf{???}}) also shows that
covariates become redundant when all lagged outcomes are included in ADH
approach. (Botosaru and Ferman 2019) show that the counterfactual
estimated by using only pre-treatment outcomes is very close to the
original ADH. (Brodersen et al. 2015a) opt to omit covariates. Finally,
both (Kinn 2018) and (Samartsidis et al. 2019) do not use covariates in
their model comparisons.

For the purpose of this paper, the argument that covariates follow the
same time varying weight structure as the outcome would be hard to
rationalize theoretically or empirically. Because of this, the
simulation opts to avoid covariates entirely.

The Monte Carlo simulation is based off of (Kinn 2018) setup. Assume the
following data generating process:

\[
\begin{aligned}
y_{j,t}(0)&=\xi_{j,t} +\psi_{j,t}+\epsilon_{j,t} & \text{j=1,..,J}\\
y_{1,t}(0)&=\sum_{j=2}^J w_{j,t}(\xi_{j,t}+\psi_{j,t})+\epsilon_{1.t}\\
\end{aligned}
\] for t=1,..,T where \(\xi_{jt}\) is the trend component, \(\psi_{jt}\)
is the seasonality component, and \(\epsilon_{jt} \sim N(0,\sigma^2)\).
Specifically, \(\xi_{jt}=c_j t+z_j\) where \(c_j,\ z_j \in \mathbb{R}\).
This will allow for each observation to have a unit-specific time
varying confounding factor and a time-invariant confounding factor.
Seasonality will be represented as
\(\psi_{j,t}=\gamma_j sin\left(\frac{\pi t}{\rho_j}\right)\). Parallel
trends are created when \(c_j=c\ \forall\ j\) and
\(\gamma_{j}=0\ \forall\ {j,t}\). The explicit data generating process
is: \[
\begin{aligned}
y_{j,t}(0)&=c_j t+z_j +\gamma_j sin\left(\frac{\pi t}{\rho_j}\right)+\epsilon_{j,t} & \text{j=2,..,J}\\
y_{1,t}(0)&=\sum_{j=2}^J w_{j,t}\left( c_j t+z_j +\gamma_j sin\left(\frac{\pi t}{\rho_j}\right) \right)+\epsilon_{1.t}\\
\end{aligned}
\] Following (Kinn 2018), a sparse set of controls will have nonzero
weights. This means properly identifying the correct controls will be
important for an accurate counterfactual. The treatment begins at period
\(T_0\). The treatment effect is initially set to 0.

This paper proposes one scenario to test continuous time varying
weights.

\hypertarget{model-testing-and-comparison}{%
\subsection{Model Testing and
Comparison}\label{model-testing-and-comparison}}

There are two components to successful inference in synthetic control:
accurate estimates of the treatment effect and accurate inference
(significant or not). I plan to test both of these by simulating
treatment effect sizes at 0\%, 0.1\%, 1\%, 10\%, and 100\% similarly to
(Brodersen et al. 2015a). These treatment effects will be calculated by
defining \(Y_{1,t}(1)=\rho Y_{0,t}(0)\) for
\(\rho \in \{1, 1.001,1.01,1.1,2\}\). For inference, I will conclude a
causal effect only if 95\% of the posteriod probability interval
excludes 0. I am purposefully not using the cummalitive effect because I
am interested in how the accuracy of the method changes over the
post-treatment period.

In order to compare TVPS state space model to ADH synthetic control
output, the median observation of the posterior distribution at each
post treatment period will be used. ADH synthetic control does not have
a confidence interval, so all comparisons must be done as point
estimates. This test will compare the recovered treatment effect size
versus the actual. Again, I will use 0\%, 0.1\%, 1\%, 10\%, and 100\%
for treatment effect sizes. I will define \(Y_{1,t}(1)\) as before.

\hypertarget{deterministic-continuous-varying-weights}{%
\subsection{Deterministic Continuous Varying
Weights}\label{deterministic-continuous-varying-weights}}

To simulate continuous varying weights, \(c_{2,t}\) and \(c_{3,t}\) are
defined .75 and .25 respectively. All other \(c_{j,t}\) are randomly
drawn from U{[}0,1{]}. In order to avoid \(y_{2,t}\) and \(y_{3,t}\)
from crossing, I set \(z_2=25\) and \(z_3=5\). I set \(\psi_{j,t}=0\)
for all j,t. Finally, I define \(w_{2,t}=.2+.6\frac{t}{T}\) and
\(w_{3,t}=1-w_{2,t}\). In order to compare to ADH, I set the sum of the
weights to unity to ensure the convex hull assumption is met in their
method.

To summarize, the parameters of this simulation are:

\begin{enumerate}
\def\labelenumi{\arabic{enumi})}
\item
  \(c_{2,t}=.75\), \(c_{3,t}=.25\), and \(c_{j,t} \sim U[0,1]\) for all
  \(j \notin \{2,3\}\)
\item
  \(z_2=25\), \(z_3=5\) and \(z_j\) is sampled from
  \(\{1,2,3,4,...,50\}\).
\item
  \(\epsilon_{j,t} \sim N(0,1)\).
\item
  T = 50, \(T_0=30\).
\item
  J = 51.
\item
  \(w_{2,t}=.2+.6\frac{t}{T}\), \(w_{3,t}=1-w_{2,t}\), and \(w_{j,t}=0\)
  for all else
\item
  \(\gamma_{j}=0\ \forall j\).
\end{enumerate}

Notice that given this setup, the data generating process can be
rewritten in recursive form:

\[
\begin{aligned}
y_{1,t}(0)&=\sum_{j=2}^J w_{j,t}\left( c_j t+z_j +\gamma_j sin\left(\frac{\pi t}{\rho_j}\right) \right)+\epsilon_{1.t}\\
w_{2,t}&=w_{2,t-1}+\frac{.6}{T}\\
w_{3,t}&=w_{2,t-1}-\frac{.6}{T}\\
w_{j,t}&=w_{j,t-1} & j\notin \{1,2,3\}\\
\end{aligned}
\] with initial conditions:

\[
\begin{aligned}
w_{2,0}&=.2\\
w_{3,0}&=.8\\
w_{j,0}&=0 & j\notin \{1,2,3\}
\end{aligned}
\]

\hypertarget{conclusion}{%
\section{Conclusion}\label{conclusion}}

This proposal adds shrinkage among time varying weights to
counterfactual analysis. The addition of shrinkage among time varying
weights will extend the scope of synthetic control to data previously
restricted from analysis. This also adds to the very limited existing
literature of state space models in counterfactual analysis. Future
research will include extending the model to multiple outcome variables
(e.g.~GDP and unemployment).

\hypertarget{work-cited-and-references}{%
\section*{Work Cited and References}\label{work-cited-and-references}}
\addcontentsline{toc}{section}{Work Cited and References}

\hypertarget{refs}{}
\leavevmode\hypertarget{ref-abadie_using_2019}{}%
Abadie, Alberto. 2019. ``Using Synthetic Controls: Feasibility, Data
Requirements, and Methodological Aspects,'' 44.

\leavevmode\hypertarget{ref-abadie_synthetic_2010}{}%
Abadie, Alberto, Alexis Diamond, and Jens Hainmueller. 2010. ``Synthetic
Control Methods for Comparative Case Studies: Estimating the Effect of
California's Tobacco Control Program.'' \emph{Journal of the American
Statistical Association} 105 (490): 493--505.
\url{https://doi.org/10.1198/jasa.2009.ap08746}.

\leavevmode\hypertarget{ref-abadie_economic_2003}{}%
Abadie, Alberto, and Javier Gardeazabal. 2003. ``The Economic Costs of
Conflict: A Case Study of the Basque Country.'' \emph{American Economic
Review} 93 (1): 113--32.
\url{https://doi.org/10.1257/000282803321455188}.

\leavevmode\hypertarget{ref-athey_matrix_2018}{}%
Athey, Susan, Mohsen Bayati, Nikolay Doudchenko, Guido Imbens, and
Khashayar Khosravi. 2018. ``Matrix Completion Methods for Causal Panel
Data Models.'' \emph{arXiv:1710.10251 {[}Econ, Math, Stat{]}},
September. \url{http://arxiv.org/abs/1710.10251}.

\leavevmode\hypertarget{ref-athey_state_2017}{}%
Athey, Susan, and Guido W. Imbens. 2017. ``The State of Applied
Econometrics: Causality and Policy Evaluation.'' \emph{Journal of
Economic Perspectives} 31 (2): 3--32.
\url{https://doi.org/10.1257/jep.31.2.3}.

\leavevmode\hypertarget{ref-belmonte_hierarchical_2014}{}%
Belmonte, Miguel A. G., Gary Koop, and Dimitris Korobilis. 2014.
``Hierarchical Shrinkage in Time-Varying Parameter Models: Hierarchical
Shrinkage in Time-Varying Parameter Models.'' \emph{Journal of
Forecasting} 33 (1): 80--94. \url{https://doi.org/10.1002/for.2276}.

\leavevmode\hypertarget{ref-bhattacharya_fast_2016}{}%
Bhattacharya, Anirban, Antik Chakraborty, and Bani K. Mallick. 2016.
``Fast Sampling with Gaussian Scale-Mixture Priors in High-Dimensional
Regression.'' \emph{arXiv:1506.04778 {[}Stat{]}}, June.
\url{http://arxiv.org/abs/1506.04778}.

\leavevmode\hypertarget{ref-billmeier_assessing_2013}{}%
Billmeier, Andreas, and Tommaso Nannicini. 2013. ``Assessing Economic
Liberalization Episodes: A Synthetic Control Approach.'' \emph{Review of
Economics and Statistics} 95 (3): 983--1001.
\url{https://doi.org/10.1162/REST_a_00324}.

\leavevmode\hypertarget{ref-bitto_achieving_2019}{}%
Bitto, Angela, and Sylvia Frühwirth-Schnatter. 2019. ``Achieving
Shrinkage in a Time-Varying Parameter Model Framework.'' \emph{Journal
of Econometrics} 210 (1): 75--97.
\url{https://doi.org/10.1016/j.jeconom.2018.11.006}.

\leavevmode\hypertarget{ref-botosaru_role_2019}{}%
Botosaru, Irene, and Bruno Ferman. 2019. ``On the Role of Covariates in
the Synthetic Control Method.'' \emph{The Econometrics Journal},
January, utz001. \url{https://doi.org/10.1093/ectj/utz001}.

\leavevmode\hypertarget{ref-brodersen_inferring_2015}{}%
Brodersen, Kay H., Fabian Gallusser, Jim Koehler, Nicolas Remy, and
Steven L. Scott. 2015a. ``Inferring Causal Impact Using Bayesian
Structural Time-Series Models.'' \emph{The Annals of Applied Statistics}
9 (1): 247--74. \url{https://doi.org/10.1214/14-AOAS788}.

\leavevmode\hypertarget{ref-brodersen_inferring_2015-1}{}%
---------. 2015b. ``Inferring Causal Impact Using Bayesian Structural
Time-Series Models.'' \emph{The Annals of Applied Statistics} 9 (1):
247--74. \url{https://doi.org/10.1214/14-AOAS788}.

\leavevmode\hypertarget{ref-campos_institutional_2019}{}%
Campos, Nauro F., Fabrizio Coricelli, and Luigi Moretti. 2019a.
``Institutional Integration and Economic Growth in Europe.''
\emph{Journal of Monetary Economics} 103 (May): 88--104.
\url{https://doi.org/10.1016/j.jmoneco.2018.08.001}.

\leavevmode\hypertarget{ref-campos_institutional_2019-1}{}%
---------. 2019b. ``Institutional Integration and Economic Growth in
Europe.'' \emph{Journal of Monetary Economics} 103 (May): 88--104.
\url{https://doi.org/10.1016/j.jmoneco.2018.08.001}.

\leavevmode\hypertarget{ref-dangl_predictive_2012}{}%
Dangl, Thomas, and Michael Halling. 2012. ``Predictive Regressions with
Time-Varying Coefficients.'' \emph{Journal of Financial Economics} 106
(1): 157--81. \url{https://doi.org/10.1016/j.jfineco.2012.04.003}.

\leavevmode\hypertarget{ref-doudchenko_balancing_2016}{}%
Doudchenko, Nikolay, and Guido Imbens. 2016. ``Balancing, Regression,
Difference-in-Differences and Synthetic Control Methods: A Synthesis.''
w22791. Cambridge, MA: National Bureau of Economic Research.
\url{https://doi.org/10.3386/w22791}.

\leavevmode\hypertarget{ref-durbin_simple_2002}{}%
Durbin, J. 2002. ``A Simple and Efficient Simulation Smoother for State
Space Time Series Analysis.'' \emph{Biometrika} 89 (3): 603--16.
\url{https://doi.org/10.1093/biomet/89.3.603}.

\leavevmode\hypertarget{ref-durbin_time_2012}{}%
Durbin, J., and S. J. Koopman. 2012. \emph{Time Series Analysis by State
Space Methods}. 2nd ed. Oxford Statistical Science Series 38. Oxford:
Oxford University Press.

\leavevmode\hypertarget{ref-fruhwirth-schnatter_stochastic_2010}{}%
Frühwirth-Schnatter, Sylvia, and Helga Wagner. 2010. ``Stochastic Model
Specification Search for Gaussian and Partial Non-Gaussian State Space
Models.'' \emph{Journal of Econometrics} 154 (1): 85--100.
\url{https://doi.org/10.1016/j.jeconom.2009.07.003}.

\leavevmode\hypertarget{ref-hastie_statistical_2015}{}%
Hastie, Trevor, Robert Tibshirani, and Martin Wainwright. 2015.
\emph{Statistical Learning with Sparsity: The Lasso and
Generalizations}. Monographs on Statistics and Applied Probability 143.
Boca Raton: CRC Press, Taylor \& Francis Group.

\leavevmode\hypertarget{ref-huber_inducing_2019}{}%
Huber, Florian, Gary Koop, and Luca Onorante. 2019. ``Inducing Sparsity
and Shrinkage in Time-Varying Parameter Models.'' \emph{arXiv:1905.10787
{[}Econ{]}}, December. \url{http://arxiv.org/abs/1905.10787}.

\leavevmode\hypertarget{ref-kapetanios_time-varying_2018}{}%
Kapetanios, George, and Filip Zikes. 2018. ``Time-Varying Lasso.''
\emph{Economics Letters} 169 (August): 1--6.
\url{https://doi.org/10.1016/j.econlet.2018.04.029}.

\leavevmode\hypertarget{ref-kaul_synthetic_2018}{}%
Kaul, Ashok, Stefan Kloßner, Gregor Pfeifer, and Manuel Schieler. 2018.
``Synthetic Control Methods: Never Use All Pre-Intervention Outcomes
Together with Covariates,'' 24.

\leavevmode\hypertarget{ref-kinn_synthetic_2018}{}%
Kinn, Daniel. 2018. ``Synthetic Control Methods and Big Data.''
\emph{arXiv:1803.00096 {[}Econ{]}}, February.
\url{http://arxiv.org/abs/1803.00096}.

\leavevmode\hypertarget{ref-makalic_high-dimensional_2016}{}%
Makalic, Enes, and Daniel F. Schmidt. 2016. ``High-Dimensional Bayesian
Regularised Regression with the BayesReg Package.''
\emph{arXiv:1611.06649 {[}Stat{]}}, December.
\url{http://arxiv.org/abs/1611.06649}.

\leavevmode\hypertarget{ref-park_bayesian_2008}{}%
Park, Trevor, and George Casella. 2008. ``The Bayesian Lasso.''
\emph{Journal of the American Statistical Association} 103 (482):
681--86. \url{https://doi.org/10.1198/016214508000000337}.

\leavevmode\hypertarget{ref-pesaran_optimal_2013}{}%
Pesaran, M. Hashem, Andreas Pick, and Mikhail Pranovich. 2013. ``Optimal
Forecasts in the Presence of Structural Breaks.'' \emph{Journal of
Econometrics} 177 (2): 134--52.
\url{https://doi.org/10.1016/j.jeconom.2013.04.002}.

\leavevmode\hypertarget{ref-polson_half-cauchy_2011}{}%
Polson, Nicholas G., and James G. Scott. 2011a. ``On the Half-Cauchy
Prior for a Global Scale Parameter.'' \emph{arXiv:1104.4937 {[}Stat{]}},
September. \url{http://arxiv.org/abs/1104.4937}.

\leavevmode\hypertarget{ref-bernardo_shrink_2011}{}%
---------. 2011b. ``Shrink Globally, Act Locally: Sparse Bayesian
Regularization and Prediction*.'' In \emph{Bayesian Statistics 9},
edited by José M. Bernardo, M. J. Bayarri, James O. Berger, A. P. Dawid,
David Heckerman, Adrian F. M. Smith, and Mike West, 501--38. Oxford
University Press.
\url{https://doi.org/10.1093/acprof:oso/9780199694587.003.0017}.

\leavevmode\hypertarget{ref-powell_imperfect_2018}{}%
Powell, David. 2018. ``Imperfect Synthetic Controls:'' 55.

\leavevmode\hypertarget{ref-samartsidis_assessing_2019}{}%
Samartsidis, Pantelis, Shaun R. Seaman, Anne M. Presanis, Matthew
Hickman, and Daniela De Angelis. 2019. ``Assessing the Causal Effect of
Binary Interventions from Observational Panel Data with Few Treated
Units.'' \emph{Statistical Science} 34 (3): 486--503.
\url{https://doi.org/10.1214/19-STS713}.

\leavevmode\hypertarget{ref-scott_predicting_nodate}{}%
Scott, Steven L, and Hal Varian. n.d. ``Predicting the Present with
Bayesian Structural Time Series,'' 21.

\leavevmode\hypertarget{ref-scott_bayesian_2013}{}%
Scott, Steven, and Hal Varian. 2013. ``Bayesian Variable Selection for
Nowcasting Economic Time Series.'' w19567. Cambridge, MA: National
Bureau of Economic Research. \url{https://doi.org/10.3386/w19567}.

\leavevmode\hypertarget{ref-noauthor_shrinkage_nodate}{}%
``Shrinkage Estimation of the Varying Coefficient Model.'' n.d., 12.

\leavevmode\hypertarget{ref-tibshirani_regression_1996}{}%
Tibshirani, Robert. 1996. ``Regression Shrinkage and Selection via the
Lasso.'' \emph{Journal of the Royal Statistical Society. Series B
(Methodological)} 58 (1): 267--88.
\url{http://www.jstor.org/stable/2346178}.

\leavevmode\hypertarget{ref-xu_generalized_2017}{}%
Xu, Yiqing. 2017. ``Generalized Synthetic Control Method: Causal
Inference with Interactive Fixed Effects Models.'' \emph{Political
Analysis} 25 (1): 57--76. \url{https://doi.org/10.1017/pan.2016.2}.

\end{document}
