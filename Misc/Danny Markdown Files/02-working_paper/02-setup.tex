% Options for packages loaded elsewhere
\PassOptionsToPackage{unicode}{hyperref}
\PassOptionsToPackage{hyphens}{url}
%
\documentclass[
]{article}
\usepackage{lmodern}
\usepackage{amssymb,amsmath}
\usepackage{ifxetex,ifluatex}
\ifnum 0\ifxetex 1\fi\ifluatex 1\fi=0 % if pdftex
  \usepackage[T1]{fontenc}
  \usepackage[utf8]{inputenc}
  \usepackage{textcomp} % provide euro and other symbols
\else % if luatex or xetex
  \usepackage{unicode-math}
  \defaultfontfeatures{Scale=MatchLowercase}
  \defaultfontfeatures[\rmfamily]{Ligatures=TeX,Scale=1}
\fi
% Use upquote if available, for straight quotes in verbatim environments
\IfFileExists{upquote.sty}{\usepackage{upquote}}{}
\IfFileExists{microtype.sty}{% use microtype if available
  \usepackage[]{microtype}
  \UseMicrotypeSet[protrusion]{basicmath} % disable protrusion for tt fonts
}{}
\makeatletter
\@ifundefined{KOMAClassName}{% if non-KOMA class
  \IfFileExists{parskip.sty}{%
    \usepackage{parskip}
  }{% else
    \setlength{\parindent}{0pt}
    \setlength{\parskip}{6pt plus 2pt minus 1pt}}
}{% if KOMA class
  \KOMAoptions{parskip=half}}
\makeatother
\usepackage{xcolor}
\IfFileExists{xurl.sty}{\usepackage{xurl}}{} % add URL line breaks if available
\IfFileExists{bookmark.sty}{\usepackage{bookmark}}{\usepackage{hyperref}}
\hypersetup{
  pdftitle={related\_work},
  hidelinks,
  pdfcreator={LaTeX via pandoc}}
\urlstyle{same} % disable monospaced font for URLs
\usepackage[margin=1in]{geometry}
\usepackage{graphicx,grffile}
\makeatletter
\def\maxwidth{\ifdim\Gin@nat@width>\linewidth\linewidth\else\Gin@nat@width\fi}
\def\maxheight{\ifdim\Gin@nat@height>\textheight\textheight\else\Gin@nat@height\fi}
\makeatother
% Scale images if necessary, so that they will not overflow the page
% margins by default, and it is still possible to overwrite the defaults
% using explicit options in \includegraphics[width, height, ...]{}
\setkeys{Gin}{width=\maxwidth,height=\maxheight,keepaspectratio}
% Set default figure placement to htbp
\makeatletter
\def\fps@figure{htbp}
\makeatother
\setlength{\emergencystretch}{3em} % prevent overfull lines
\providecommand{\tightlist}{%
  \setlength{\itemsep}{0pt}\setlength{\parskip}{0pt}}
\setcounter{secnumdepth}{-\maxdimen} % remove section numbering
\newcommand{\magenta}[1]{\textcolor{magenta}{#1}}

\title{related\_work}
\author{}
\date{\vspace{-2.5em}}

\begin{document}
\maketitle

\hypertarget{setup}{%
\section{Setup}\label{setup}}

\hypertarget{potential-outcomes-and-parameter-of-interest}{%
\subsection{Potential Outcomes and Parameter of
Interest}\label{potential-outcomes-and-parameter-of-interest}}

I define the treatment as an intervention occurring at the aggregate
level of a society. Once treated, the unit remains treated indefinitely.
The treatment status is known for all units in all time periods.

Let \(\left(Y_{j,t}(0),Y_{j,t}(1)\right)\) represent potential outcomes
in the presence and absence of a treatment with
\(t=1,...,T_{0}-1,T_0,T_{0}+1,...T\) and \(j=0,1,...,J\). Denote the
period of intervention at \(T_0\). Define the treatment status as
\(D_j=\{0,1\}\), where a 1 indicates if the unit is treated in any
period. I assume the potential outcomes are random.

Define \(Y_{j,t}=(1-D_{j,t})Y_{j,t}(0)+D_{j,t}Y_{j,t}(1)\) where
\(D_{j,t}=D_jI(t \ge T_0)\). The researcher observes the following:

\begin{align}
Y_{j,t}= \begin{cases}
Y_{j,t}(1) & D_{j,t}=1\\
Y_{j,t}(0) & D_{j,t}=0\\
\end{cases}
\end{align}

The average treatment effect of the treated unit at each period \(t\) is
denoted
\(\tau_{t}=\mathbb{E}[Y_{j,t}(1)|D_j=1]-\mathbb{E}[Y_{j,t}(0)|D_j=1]\).
In order to draw causal inference, I assume conditional independence on
past outcomes:

\begin{assumption}
Conditional Independence on Past Observed Outcomes
\end{assumption}

\begin{align}
\left\{Y_{j,T_0+i}(0),Y_{j,T_0+i}(1)\right\} \indep D_{j,T_0+i} | Y_{j,1},\dots,Y_{j,T_0}
\label{eq:ass_2}
\end{align}

for \(i \in \{1,\dots, T-T_0\}\).

The conditional independence assumption uses the full set of
pretreatment outcomes to proxy for the unobserved confounders. \textbf{A
better fit on the pretreatment outcomes implies a better representation
of the unobserved confounders and better prediction of the unobserved
potential outcome.
\textcolor{magenta}{think about getting rid of this. See Antoine's comment.}}

Since we are only interested in the treatment effect post intervention,
replace t with \(T_0+i\). The estimand can be rewritten as:

\[
\begin{aligned}
\tau_{T_0+i}&=\mathbb{E}[Y_{j,T_0+i}(1)|D_j=1]-\mathbb{E}[Y_{j,T_0+i}(0)|D_j=1]\\
&= \mathbb{E}[Y_{j,T_0+i}(1)|D_j=1]-\mathbb{E}[\mathbb{E}[Y_{j,T_0+i}(0)| Y_{j,1},\dots,Y_{j,T_0}, D_j=1 ]|D_j=1]\\
&= \mathbb{E}[Y_{j,T_0+i}(1)|D_j=1]-\mathbb{E}[\mathbb{E}[Y_{j,T_0+i}(0)| Y_{j,1},\dots,Y_{j,T_0}, D_j=0 ]|D_j=1]\\
&=\mathbb{E}[Y_{j,T_0+i}(1)|D_j=1]-\mathbb{E}[g_{T_0+i}\left(Y_{j,1},\dots,Y_{j,T_0}\right)|D_j=1]
\end{aligned}
\]

where
\(g_{T_0+i}\left(Y_{j,1}(0),\dots,Y_{j,T_0}(0)\right)=\mathbb{E}[Y_{j,T_0+i}(0)| Y_{j,1},\dots,Y_{j,T_0}, D_j=0 ]\).

Suppose only one unit, \(j=0\), is treated beginning in period \(T_0\)
and remains treated for all \(T_0+i \ge T_0\). The other J units are
unaffected by the treatment and there is no anticipation effects (SUTVA
holds, @rubin\_formal\_1990)\footnote{This is a common assumption in the
  synthetic control literature. Recently, @grossi\_synthetic\_2020 have
  introduced spillover effects in analyzing new light rail transit.}. We
observe a sample
\(\left\{\left(y_{j,1},\dots,y_{j,T},d_j\right)\right\}_{j=0}^J\) from
the distribution \(\left(Y_{j,1},\dots,Y_{j,T},D_j\right)\). Since there
is only one treated unit, \(\mathbb{E}[Y_{j,T_0+i}(1)|D_j=1]\) can be
estimated with the realization of the data, \(y_{0,T_0+i}\), and
\(\mathbb{E}[g_{T_0+i}\left(Y_{j,1},\dots,Y_{j,T_0}\right)|D_j=1]\) can
be estimated as \(g_{T_0+i}\left(y_{0,1}(0),\dots,y_{0,T_0}(0)\right)\).

The goal is to estimate the sample ATT:

\begin{align}
\hat{\tau}_{T_0+i}=y_{0,T_0+i}(1)-\hat{g}_{T_0+i}\left(y_{0,1}(0),\dots,y_{0,T_0}(0)\right)
\end{align}

\hypertarget{general-model}{%
\subsection{General Model}\label{general-model}}

There are three sources of information to estimate
\(y_{0,T_0+i}(0)={g}_{T_0+i}\left(y_{0,1}(0),\dots,y_{0,T_0}(0)\right)\):

\begin{enumerate}
\def\labelenumi{\roman{enumi})}
\item
  Untreated pre-treatment units: \textbf{y}.
\item
  Treated pre-treatment units:
  \(\mathbf{y_0}=\left(y_{0,1}(0),\dots , y_{0,T_0-1}(0)\right)\).
\item
  Untreated post-treatment units.
\end{enumerate}

Creating the counterfactual relies on utilizing all three sources of
information. The parameters are estimated in the pre-period and then
used to forecast out in the post-period. I propose using a time-varying
coefficient model. Namely:

\begin{align}
g_{T_0+i}\left(y_{0,1}(0),\dots,y_{0,T_0}(0)\right)=\sum_{j=1}^{J+1}\beta_{j,T_0+i}y_{j,T_0+i}(0)
\label{eq:mod1}
\end{align}

where \(y_{J+1,T_0+i}(0)=1\) for all t (intercept). With a time varying
structure, a perfect fit can be made for each \(y_{0,t}(0)\). More so,
there exists an infinite combination of perfect matches. In addition,
the model would have no out of sample predictive ability. The perfect
fit would have no differentiation between signal and noise. To account
for this, I add structure to the time varying coefficients through state
space modeling. The coefficients are modeled as random walks.
Incorporating \eqref{eq:mod1} into a state space framework and adding
the random walk component yields:

\begin{align}
g_{T_0+i}\left(y_{0,1}(0),\dots,y_{0,T_0}(0)\right)&= \sum_{j=1}^{J+1} \beta_{j,t}y_{j,t}(0)+\epsilon_t & \epsilon_t \sim \mathcal{N}(0, \sigma^2) &\label{eq:rw1}\\
\beta_{j,t}&=\beta_{j,t-1}+\eta_{j,t} & \eta_{j,t} \sim \mathcal{N}(0,\theta_j) &\ \ \ \  \forall j \label{eq:rw}\\
\beta_{j,0}&\sim \mathcal{N}(\beta_j,\theta_j P_{jj}) &\ \ \ \ \forall j \label{eq:rw2}
\end{align}

It is common in state space literature to model time varying parameters
as random walks {[}@dangl\_predictive\_2012;
@belmonte\_hierarchical\_2014; @bitto\_achieving\_2019{]}. The choice of
random walk allows for a useful decomposition of the coefficients into
time-varying and constant components.

The parameters of the model are
\(\mathbf{v}=\{\sigma^2, \beta_1,...\beta_{J+1},\theta_1,...\theta_{J+1}\}\)
with \(\epsilon_t\) and \(\eta_{j,t}\) assumed independent of all other
unknowns. \(P_{jj}\) is a hyperparameter set to ensure the initial
distribution is disperse. In practice, \(P_{jj}\) is set to a very large
value creating a diffuse initial state {[}@durbin\_time\_2012{]}.

\end{document}
