% Options for packages loaded elsewhere
\PassOptionsToPackage{unicode}{hyperref}
\PassOptionsToPackage{hyphens}{url}
%
\documentclass[
]{article}
\usepackage{lmodern}
\usepackage{amssymb,amsmath}
\usepackage{ifxetex,ifluatex}
\ifnum 0\ifxetex 1\fi\ifluatex 1\fi=0 % if pdftex
  \usepackage[T1]{fontenc}
  \usepackage[utf8]{inputenc}
  \usepackage{textcomp} % provide euro and other symbols
\else % if luatex or xetex
  \usepackage{unicode-math}
  \defaultfontfeatures{Scale=MatchLowercase}
  \defaultfontfeatures[\rmfamily]{Ligatures=TeX,Scale=1}
\fi
% Use upquote if available, for straight quotes in verbatim environments
\IfFileExists{upquote.sty}{\usepackage{upquote}}{}
\IfFileExists{microtype.sty}{% use microtype if available
  \usepackage[]{microtype}
  \UseMicrotypeSet[protrusion]{basicmath} % disable protrusion for tt fonts
}{}
\makeatletter
\@ifundefined{KOMAClassName}{% if non-KOMA class
  \IfFileExists{parskip.sty}{%
    \usepackage{parskip}
  }{% else
    \setlength{\parindent}{0pt}
    \setlength{\parskip}{6pt plus 2pt minus 1pt}}
}{% if KOMA class
  \KOMAoptions{parskip=half}}
\makeatother
\usepackage{xcolor}
\IfFileExists{xurl.sty}{\usepackage{xurl}}{} % add URL line breaks if available
\IfFileExists{bookmark.sty}{\usepackage{bookmark}}{\usepackage{hyperref}}
\hypersetup{
  pdftitle={intro},
  hidelinks,
  pdfcreator={LaTeX via pandoc}}
\urlstyle{same} % disable monospaced font for URLs
\usepackage[margin=1in]{geometry}
\usepackage{graphicx,grffile}
\makeatletter
\def\maxwidth{\ifdim\Gin@nat@width>\linewidth\linewidth\else\Gin@nat@width\fi}
\def\maxheight{\ifdim\Gin@nat@height>\textheight\textheight\else\Gin@nat@height\fi}
\makeatother
% Scale images if necessary, so that they will not overflow the page
% margins by default, and it is still possible to overwrite the defaults
% using explicit options in \includegraphics[width, height, ...]{}
\setkeys{Gin}{width=\maxwidth,height=\maxheight,keepaspectratio}
% Set default figure placement to htbp
\makeatletter
\def\fps@figure{htbp}
\makeatother
\setlength{\emergencystretch}{3em} % prevent overfull lines
\providecommand{\tightlist}{%
  \setlength{\itemsep}{0pt}\setlength{\parskip}{0pt}}
\setcounter{secnumdepth}{-\maxdimen} % remove section numbering

\title{intro}
\author{}
\date{\vspace{-2.5em}}

\begin{document}
\maketitle

\hypertarget{introduction}{%
\section{Introduction}\label{introduction}}

\magenta{In this paper, I consider the problem of estimating the causal effect of an intervention on an outcome of interest when there is one unit affected by the intervention and the relationship between the unit of interest and all other units may be non-constant. 

A common approach to this problem is the synthetic control framework. The goal is to construct a counterfactual for the treated unit as a linear combination of untreated units. This approach has been employed by practitioners in all aspects of economics including (but not limited to) the effects of terrorism [@abadie_economic_2003], trade policies [@billmeier_assessing_2013], natural disasters [@cavallo_catastrophic_2013] and social issues [@powell_imperfect_2018]. 

@abadie_synthetic_2010 show if there exists some linear combination of untreated units such that a perfect pretreatment estimate of the treated unit exists, then the asymptotic bias of the treatment effect is zero. However, they explicitly warn against the uses of synthetic control when an accurate counterfactual cannot be constructed. One indicator of a poorly constructed counterfactual is poor fit in the pretreatment period. The poor pretreatment fit is a sign the linear combination of untreated units is not properly representing the unobserved confounders.}

This paper proposes incorporating time varying coefficients. An
immediate concern with time varying coefficients is the risk of
overfitting. Recent advances in macroeconometric forecasting have
developed methods to address this concern {[}@dangl\_predictive\_2012;
@bitto\_achieving\_2019; @belmonte\_hierarchical\_2014{]}. The methods
rely on two ideas: non-centered state space modeling and Bayesian
shrinkage. Non-centered state space modeling decomposes time varying
parameters into a time varying component and a time invariant component.
By decomposing each time varying parameter, Bayesian shrinkage
techniques can be performed to ``shrink'' irrelevant parameters towards
zero increasing out of sample performance. These two techniques allow
the proposed model to perform as well as a static-coefficient model when
the true data generating process involves only static coefficients and
superior otherwise.

The key contribution of this work is an alternative application of time
varying parameters to the synthetic control problem. First, I propose a
time varying parameter model based on recent macroeconometric advances
to estimate the counterfactual. Second, I compare a popular state space
counterfactual model, @brodersen\_inferring\_2015, to the proposed model
in simulation studies. Prior to this paper, testing of this class of
model focused on high frequency data including stock prices and
inflation rates {[}@dangl\_predictive\_2012; @bitto\_achieving\_2019{]}.
This differs greatly to a synthetic control setting where data tends to
be yearly or monthly with few pre and post periods and potentially many
controls. Finally, I compare the model's performance to
@brodersen\_inferring\_2015 and @abadie\_synthetic\_2010 using the
German Reunification {[}@abadie\_comparative\_2015{]} and the California
Tobacco tax {[}@abadie\_synthetic\_2010{]} case studies.

\hypertarget{related-work}{%
\subsection{Related Work}\label{related-work}}

Developments to synthetic control can be summarized into three main
categories: i) multiple treatments/outcomes {[}@xu\_generalized\_2017;
@athey\_matrix\_2020; @lhour\_penalized\_2019{]}, ii) inference
{[}@li\_statistical\_2019; @cattaneo\_prediction\_2019;
@chernozhukov\_exact\_2019{]}, and iii) counterfactual estimation. This
paper is focused on counterfactual estimation. For a full review of
synthetic controls, the reader is directed to @abadie\_using\_2019. For
an in depth comparison of multiple synthetic control approaches, the
reader is directed to @samartsidis\_assessing\_2019 and
@kinn\_synthetic\_2018.

An increasingly common issue in the case study literature is more
untreated units than pre-treatment periods. For example,
@abadie\_synthetic\_2010 considers a situation in which there are 29
untreated units and 17 pretreatment periods. Past researchers have
addressed this issue with machine learning techniques
(e.g.~@doudchenko\_balancing\_2016, @athey\_matrix\_2018).
@pang\_modeling\_2010 recommended a model comparison algorithm using
Bayes factors while @pang\_bayesian\_2020 and
@brodersen\_inferring\_2015 incorporate shrinkage priors. These methods
place a majority of mass of the prior distribution at zero. This forces
coefficients biased towards zero which allows for the usage of more
covariates than observations and better out of sample predictions while
avoiding overfitting.

Bayesian methods have been used to estimate the counterfactual in a
synthetic control setting. @brodersen\_inferring\_2015 models the
counterfactual using a combination of spike and slab priors and linear
Gaussian state space modeling. The spike and slab priors are used to
perform automatic variable selection. The authors allow for the
coefficients to be constant or dynamic. However, they warn of the
dangers of overfitting and implausibly large probability intervals with
dynamic coefficients {[}@brodersen\_inferring\_2015{]}.

This paper aims to solve the issues @brodersen\_inferring\_2015 faced
when incorporating dynamic coefficients. First, the proposed model
incorporates the decomposition of time varying coefficients. Second, the
model uses a different set of priors to create the Bayesian LASSO. The
decomposition paired with the choice of priors solves the issues of
overfitting and implausibly large probability intervals. The proposed
model allows for shrinkage on the time varying and time invariant
portion of the coefficient. Adding the parameter decomposition and
Bayesian Lasso allows for the use of time varying parameters without
implausibly large probability intervals.

@pang\_bayesian\_2020 develop a Bayesian approach based off of the
@xu\_generalized\_2017 linear factors model. Utilizing
@bitto\_achieving\_2019 non-centered parameterization,
@pang\_bayesian\_2020 explicitly model the latent factors as well as
allowing for time varying coefficients. Similar to this paper, they
employ a state-space Bayesian framework with shrinkage priors. This
paper differs from @pang\_bayesian\_2020 in several key ways. The
proposed model begins with the established non-centered parameterization
framework and adds causal assumptions while @pang\_bayesian\_2020 begins
with the linear factors models and incorporates time varying components.
The different initial points leads to vastly different functional forms.
I focus on the case where there is one treated unit and the only
predictors are untreated units. @pang\_bayesian\_2020 extend their model
to multiple treated units utilizing additional covariates. The final
difference between the two papers is scope and purpose.
@pang\_bayesian\_2020 establish a full Bayesian framework to the
synthetic control setting. The purpose of this paper is to introduce
time varying parameters to the synthetic control framework and
investigate their usefulness. It is not entirely clear the increased
flexibility of this model will be beneficial in synthetic control
settings given the small sample size. In this manner, the papers can be
seen as compliments to one another.

\end{document}
